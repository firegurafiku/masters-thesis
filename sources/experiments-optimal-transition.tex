
\section{Теория плавных переходов}


\subsection{Общие соотношения}

Плавные переходы применяются для согласования линий с различными волновыми
сопротивлениями. Их выполняют в виде отрезка неоднородной линии с плавно
меняющимся волновым сопротивлением~$W$, включенного между регулярными линиями
с волновыми сопротивлениями~$W_1$ и~$W_2$. По сравнению со ступенчатыми
переходами плавные переходы отличаются большей широкополосностью, большей
электрической прочностью и, как правило, значительно менее жесткими требованиями
к точности изготовления. Однако длина плавного перехода при том же допуске
на рассогласование больше, чем длина ступенчатого перехода.

В антенной технике плавные переходы появились раньше ступенчатых, однако анализ
плавного перехода удобно строить, рассматривая его как предельный случай
ступенчатого перехода при неограниченном уменьшении длины ступеньки
и соответствующего ей скачка волновых сопротивлений. При малой длине
ступеньки~$\Delta{z}$ и малом скачке волновых сопротивлений~$\Delta{W}$
коэффициент отражения
\begin{equation*}
    \rho \approx \frac{\Delta{W}}{2W}
         \approx \frac{\Delta{z}}{2W}\frac{dW}{dz}
         = \frac{1}{2} (\ln W)' \Delta{z}.
\end{equation*}

Функцию $\frac{1}{2W}\frac{dW}{dz}$, часто называемую в литературе
\emph{функцией местных отражений}, обозначим~$N(z)$. Эта функция удовлетворяет
условию
\begin{equation}
    \label{eq:GradingTransition:ConditionForN}
    \int\limits_0^L N(z)dz
        = \int\limits_0^L \frac{1}{2} (\ln W)' dz
        = \frac{1}{2} \ln \frac{W_2}{W_1},
\end{equation}
где $L$ --- длина перехода.

Связь между прямыми и обратными волнами в линии получают в виде системы
дифференциальных уравнений
\begin{equation}
    \label{eq:GradingTransition:MainDiffEquations}
    \frac{d}{dz}
    \begin{pmatrix}
        u_{пр} \\
        u_{обр}
    \end{pmatrix} =
    \begin{pmatrix}
        -i\beta & -N(z) \\
        -N(z)   & i\beta
    \end{pmatrix}
    \begin{pmatrix}
        u_{пр} \\
        u_{обр}
    \end{pmatrix},
\end{equation}
где $\beta=2\pi/\lambda$ --- постоянная распространения.

В общем случае произвольной зависимости~$N(z)$ не существует явного
аналитического решения этой системы, однако, если отражение от перехода мало,
решение может быть записало в виде ряда последовательных приближений. Для этого
система заменяется эквивалентной системой интегральных уравнений:
\begin{align}
    \label{eq:GradingTransition:MainIntEquations}
    u_{пр}(z) &= u_{пр}(0) e^{-j\beta z} -
        \int\limits_0^z N(x) u_{обр}(x) e^{-j\beta (z-x)} dx, \\
    u_{обр}(z) &= u_{обр}(L) e^{-j\beta (L-z)} +
        \int\limits_0^z N(x) u_{пр}(x) e^{j\beta (z-x)} dx.
\end{align}

Пусть падающая волна~$u_{пр}$ во входном течении перехода $z=0$
%(см.~рис.~\ref{fig:GradingTransition:Overview})
равна 1, a в выходном сечении
$z=L$ отсутствует отраженная волна~$u_{обр}$. В первом приближении,
соответствующем приближению однократного рассеяния, не учитывается вторичное
преобразование обратной волны в прямую. В этом приближении:
\begin{align}
    \label{eq:GradingTransition:FirstApproximation}
    u_{пр}(z)  &= e^{-j\beta z}, \\
    u_{обр}(z) &= \int\limits_z^L N(x) u_{пр}(x) e^{-j\beta(x-z)} dx.
\end{align}

Найденное значение~$u_{обр}(z)$ подставляется в первую строку \eqref{eq:GradingTransition:MainIntEquations}, и
вычисленное уточненное значение $u_{пр}(z)$ подставляется во вторую строку \eqref{eq:GradingTransition:MainIntEquations}.
Процедура уточнения может повторяться, и каждое последующее приближение
определяет добавку к прямой и отраженной волнам. Практический интерес, однако,
представляют лишь те случаи, когда отраженная переходом волна мала и приближение
однократного рассеяния справедливо. В этом приближении коэффициент отражения
перехода
\begin{equation}
    \label{eq:GradingTransition:FirstApproximationReflectionCoeff}
    \Gamma(\beta)
        = u_{}(0)
        = \int\limits_0^L N(z) e^{-2j\beta z} dz.
\end{equation}

Это выражение по виду совпадает с выражением, описывающим диаграмму
направленности антенны с линейным раскрывом. Роль амплитудного распределения
играет функция местных отражений~$N(z)$, а зависимость~$\Gamma$ от частотной
переменной аналогична зависимости диаграммы направленности от угла. С ростом
коэффициент отражения имеет осциллирующий вид, аналогичный лепесткам диаграммы
направленности, и уменьшение отражения вследствие применения плавного перехода
имеет тот же смысл, что и уменьшение излучения в области боковых лепестков
(по сравнению с главным лучом). Ввиду этой аналогии все известные соображения
о характере влияния амплитудного распределения на уровень боковых лепестков
применимы для оценки влияния распределения~$N(z)$ на получаемой уровень
согласования. В частности, коэффициент отражения тем меньше, чем сильнее~$N(z)$
спадает к краям перехода.
При плавной зависимости~$N(z)$ выражение \eqref{eq:GradingTransition:FirstApproximationReflectionCoeff} может быть разложено
в асимптотический ряд путем применения формулы интегрирования по частям:
\begin{align*}
    \Gamma
    &= \int\limits_0^L \frac{N(z)}{-2j\beta}
       \left( -2j \beta z e^{-2j\beta z} \right) dz \\
    &= \left.\frac{N(z)}{-2j\beta} e^{-2j\beta z} \right|_0^L -
       \int\limits_0^L \frac{N'(z)}{-2j\beta} e^{-2j\beta z} dz, \\
    &= \left.\frac{N(z)}{-2j\beta} e^{-2j\beta z} \right|_0^L -
       \left.\frac{N'(z)}{(2j\beta)^2} e^{-2j\beta z} \right|_0^L +
       \int\limits_0^L \frac{N''(z)}{(2j\beta)^2} e^{-2j\beta z} dz.
\end{align*}

Многократное применение этой формулы дает разложение
\begin{multline}
    \label{eq:GradingTransition:GammaSeries}
    \Gamma =
        -j\frac{N(0)}{2\beta}
        -\frac{N'(0)}{(2\beta)^2}
        +j\frac{N''(0)}{(2\beta)^2}
        +j\frac{N''(0)}{(2\beta)^3}
        + \ldots \\
        + \left[
            j\frac{N(L)}{2\beta}
            +\frac{N'(L)}{(2\beta)^2}
            +j\frac{N''(L)}{(2\beta)^3}
            +\ldots
        \right] e^{2j\beta L}
        + \int\limits_0^L \frac{N^{(k)}(z)}{(2j\beta)^k} e^{-2j\beta z}dz.
\end{multline}

Это разложение является асимптотическим разложением по обратным степеням
большого параметра; поэтому, начиная с некоторого члена, оно расходится, и его
следует продолжать лишь до тех пор, пока его члены убывают. При этом
интегральный член меньше последнего члена разложения  и им можно пренебречь.

Как видно из разложения \eqref{eq:GradingTransition:GammaSeries}, коэффициент отражения в основном определяется
поведением~$N(z)$ на краях перехода. Именно в этих областях требуется достаточно
точное выполнение его профиля. В средней же части перехода отклонения от
требуемого профиля почти не сказываются на уровне согласования, что и определяет
сравнительно слабые требования к точности изготовления.

Из выражения \eqref{eq:GradingTransition:GammaSeries} следует также, что чем более плавно сопрягается переход
с регулярной линией, тем меньше коэффициент отражения, тем быстрее он убывает
с частотой. Если  на концах перехода коэффициент отражения в основном
определяется первым членом разложения:
\begin{equation}
    \label{eq:GradingTransition:GammaFirstComponent}
    \Gamma \approx \frac{1}{2j\beta}
    \left[
        N(0) - N(L) e^{-2j\beta L}
    \right],
\end{equation}
то с увеличением частоты амплитуда осцилляции коэффициента отражения убывает
обратно пропорционально~$\beta$, т.~е. частоте. Если на концах перехода значения
$dW(z)/dz$ и, следовательно, $N(z)$ равны нулю, разложение начинается со второго
члена:
\begin{equation*}
    \Gamma \approx - \frac{j}{(2\beta)^2}
    \left[
        N'(0) - N'(L) e^{-2j\beta L}
    \right].
\end{equation*}

В этом случае осцилляции~$\Gamma$ убывают значительно быстрее, обратно
пропорционально~$\beta^2$, и т.~д. Приведенные асимптотические выражения
справедливы для достаточно больших значений~$\beta$. Из условия (1.1) вытекают
следующие оценки:
\begin{gather*}
    NL \sim \frac{1}{2} \ln\frac{W_2}{W_1} \sim 1, \\
    \frac{(N'L)L}{2} \sim NL \sim 1,
\end{gather*}
и разложение справедливо при $2\beta L \le 2\pi$ или~$L \le \lambda/2$, т.~е.
в области боковых лепестков.

Далее рассмотрим конкретные виды плавных переходов.


\subsection{Экспоненциальный переход}

В этом наиболее простом типе перехода $N(z)=const=N_0$ и волновое
сопротивление изменяется по закону $W(z)= - W_1 \exp (2 N_0 z)$, с чем связано
его название. Как следует из (1.1), $N_0 = (1/2L) \ln (W_2/W_l)$. При постоянном
значении~$N$ система~\eqref{eq:GradingTransition:MainDiffEquations} имеет явное
аналитическое решение. Подставляя в~\eqref{eq:GradingTransition:MainDiffEquations}
\begin{align*}
    u_{пр}(z)  &= a e^{-j\beta_1 z}, \\
    u_{обр}(z) &= b e^{-j\beta_1 z},
\end{align*}
получаем систему линейных уравнений относительно~$а$ и~$b$:
\begin{align*}
    ja(\beta_1 - \beta) - N_0\beta &= 0, \\
    ja(\beta_1 + \beta) - N_0 a &= 0.
\end{align*}

Условием существования ненулевого решения является равенство нулю определителя
этой системы: $\beta_1^2–\beta^2+N_0^2=0$, откуда
\begin{align*}
    \beta_1 &= \pm \sqrt{\beta^2 - N_0^2}, \\
    \beta   &= j a N_0 \left( \beta \pm \sqrt{\beta^2-N_0^2} \right).
\end{align*}

Общее решение имеет вид:
\begin{align*}
    u_{пр}(z)  &= a_1 e^{-j\beta_1 z} + a_2 e^{j\beta_1 z}, \\
    u_{обр}(z) &= -\frac{j}{N}
    \left[
        a_1 \left( \beta + \sqrt{\beta^2-N_0^2} \right) e^{-j\beta_1 z}
    \right. + \\ &+
    \left.
        a_2 \left( \beta - \sqrt{\beta^2-N_0^2} \right) e^{j\beta_1 z}
    \right].
\end{align*}

Из граничных условий $u_{пр}(0)=1$ и $u_{обр}(L)=0$ получаем
\begin{align*}
    a_1 &=
    \left[
        1 - \frac{\beta-\sqrt{\beta^2-N_0^2}}{\beta+\sqrt{\beta^2-N_0^2}}
        e^{-2j\beta_1 L}
    \right]^{-1}, \\
    a_2 &=
    \left[
        1 - \frac{\beta+\sqrt{\beta^2-N_0^2}}{\beta-\sqrt{\beta^2-N_0^2}}
        e^{2j\beta_1 L}
    \right]^{-1}.
\end{align*}

Коэффициент отражения~$u_{обр}(0)$ находится как
\begin{equation}
    \label{eq:GradingTransition:ReflectionCoeff}
    \Gamma = -j N
        \frac{1-e^{-2jL\sqrt{\beta^2-N_0^2}}}{\beta+\sqrt{\beta^2-N_0^2}}
        \left[ 1 -
            \frac{\beta-\sqrt{\beta^2-N_0^2}}{\beta+\sqrt{\beta^2-N_0^2}}
            e^{-2jL\sqrt{\beta^2-N_0^2}}
        \right]^{-1}.
\end{equation}

При больших~$\beta$~\eqref{eq:GradingTransition:ReflectionCoeff} стремится
к приближенному выражению~\eqref{eq:GradingTransition:GammaFirstComponent}, которое
в данном случае точно соответствует формуле~\eqref{eq:GradingTransition:FirstApproximation}. Уже при
$\beta L=2 N_0 L=\ln (W_2/W_1)$ отличие становится пренебрежимо малым.
При $\beta L=0$ точное выражение~~\eqref{eq:GradingTransition:ReflectionCoeff} равно
\begin{equation*}
    -\frac{1-e^{2 N_0 L}}{1+e^{2 N_0 L}} =
     \frac{W_2-W_1}{W_2-W_1},
\end{equation*}
тогда как приближенное выражение дает несколько большую величину
$\Gamma = NL = \ln\sqrt{W_2/W_1}$, которая, впрочем, при встречающихся
на практике перепадах волновых сопротивлений не слишком отличается от точной.
Это сравнение еще раз подтверждает приемлемую точность приближения однократного
рассеяния.

Экспоненциальный переход является наиболее коротким при невысоких требованиях
к согласованию, аналогично тому, как ширина главного лепестка ДН антенны
наименьшая при равномерном амплитудном распределении. Первый выброс коэффициента
отражения, соответствующий первому лепестку диаграммы направленности, имеет
место при $\beta L=\val{1.43}$ и равен $\val{0.22}\ln\sqrt{W_2/W_1}$. Этот
выброс является наибольшим, и начиная с $\beta L=\val{0.8}\pi$ коэффициент
отражения не превышает этого значения.

Если же требуется обеспечить более высокий уровень согласования, необходимо
существенное увеличение длины перехода, поскольку, как указывалось выше, выбросы
коэффициента отражения с ростам~$\beta L$ убывают медленно. Например,
коэффициент отражения, равный $\val{0.7}\ln\sqrt{W_2/W_1}$ получается лишь после
четвертого всплеска характеристики, т.~е. $\beta L$ должно быть больше
$\val{4.5}\pi$ ($L>\val{2.25}\lambda$).


\subsection{Оптимальные переходы}

При специально подобранных законах изменения~$W(z)$ можно получить достаточно
высокое согласование при небольшой длине перехода. Как указывалось выше,
улучшение согласования может быть достигнуто путем более гладкого сопряжения
перехода с регулярными линиями так, чтобы разложение \eqref{eq:GradingTransition:GammaSeries} начиналось с более
высоких членов. Например, при~$N(z)$ вида $C\sin\frac{\pi z}{L}$, где
\begin{equation*}
    C = \frac{\pi}{\eta L} \ln \frac{W_2}{W_1},
\end{equation*}
из \eqref{eq:GradingTransition:FirstApproximationReflectionCoeff} следует
\begin{equation*}
    \Gamma = \ln \sqrt{\frac{W_2}{W_1}} e^{-j\beta L} \cos{\beta L}
    \left[
        1 - \left( \frac{2\beta L}{\pi} \right) ^2
    \right]^{-1}.
\end{equation*}

%График модуля этой функции показан на рис.~\ref{fig:GradingTransition:GammaFunction}.
Максимальное значение выброса коэффициента отражения равно
$\val{0.07} \cdot \ln\sqrt{W_2/W_1}$, и начиная со значения $\beta L=\val{1.35}\pi$
($L=\val{0.68}\lambda$) коэффициент отражения не превышает этого значения.
Еще более высокое согласование можно получить, если подбирать~$W(z)$ таким
образом, чтобы не равные нулю члены разложения \eqref{eq:GradingTransition:GammaSeries} компенсировали друг друга.
Оптимальная равноколебательная характеристика, соответствующая минимальной длине
перехода при заданном уровне согласования, получается в предельном случае
чебышевского ступенчатого перехода при стремлении к бесконечности верхней
границы полосы пропускания. При этом средняя длина волны, а с ней и длина
ступеньки стремятся к нулю, число ступенек стремится к бесконечности, а общая
длина перехода стремится к конечному пределу. Длина такого перехода и допуск
на рассогласование связаны соотношением
\begin{equation*}
    \frac{|\Gamma|_{макс}}{\ln\sqrt{\frac{W_2}{W_1}}} =
    \frac{1}{\ch \beta L}.
\end{equation*}

Из того факта, что чебышевскнй плавный переход является предельным случаем
ступенчатого перехода при бесконечном расширении полосы пропускания, следует,
что при любой конечной рабочей полосе длина соответствующего ступенчатого
перехода меньше длины плавного перехода.

Отличительной особенностью чебышевского плавного перехода является наличие
скачков волнового сопротивления на его концах. Необходимость таких скачков
видна из разложения \eqref{eq:GradingTransition:GammaSeries}, все члены которого убывают с ростом частоты, тогда
как чебышевской характеристике соответствуют неубывающие осцилляции. Коэффициент
отражения от каждого из скачков~$\Gamma_{ск}$ связан с~$|\Gamma|_{макс}$ очевидным
соотношением $\Gamma_{cк}=|\Gamma|_{макс}/2$, вытекающим из поведения $|\Gamma|$ при
$\beta L \to \infty$.

На концах перехода~$N$, $N''$,~\dots равны нулю, а $N'$, $N'''$,~\dots подбирают таким
образом, чтобы скомпенсировать~$\Gamma_{ск}$ при умеренно больших значениях~$\beta L$.
Наиболее простой практический способ нахождения закона изменения~$W(z)$
заключается в расчете ступенчатого перехода с достаточно большим числом ступенек.
Поскольку чебышевский переход не является собственно плавным, что снижает его
практическую ценность, представляет интерес нахождение распределений~$N(z)$,
позволяющих получить характеристику, близкую к чебышевской, но без скачков
волновых сопротивлений. Подобная характеристика может быть получена, если на
концах перехода~$N$, $N''$,~\dots равны нулю, а члены, содержащие $N'$, $N'''$,~\dots,
подобраны таким образом, чтобы величина первых всплесков характеристики была
одинаковой и минимально возможной. Наиболее простой закон распределения~$N(z)$,
позволяющий это сделать, имеет вид
\begin{equation*}
    N(z) - N_0 \left( 1 - C \cos \frac{2\pi z}{L} \right).
\end{equation*}

Подставляя это выражение, получаем
\begin{equation*}
    \Gamma = \left(
        \frac{\sin \beta L}{\beta L} -
        C\frac{\beta L \sin \beta L}{(\beta L)^2 - L^2}
    \right)
    e^{-j\beta L}
    \ln\sqrt{\frac{W_2}{W_1}}.
\end{equation*}

Коэффициент~$С$ может быть подобран, например, таким образом, чтобы всплески
характеристики в интервалах $(\pi; 2\pi)$ и $(2\pi; З\pi)$ имели одинаковое
значение. Это соответствует $С=\val{0.63}$.

Значительно более высокое согласование получается в том случае, если
минимизируются всплески в интервалах $(2\pi; 3\pi)$ и $(З\pi; 4\pi)$.
