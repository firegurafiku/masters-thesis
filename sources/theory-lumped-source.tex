%
%
%

\section{Точечный резистивный источник напряжения}
\label{div:LumpedSource}

Возможность моделировать сосредоточенные элементы вводится как расширение
основного метода~\cite{bib:Taflove1995,bib:Davidson2005,bib:PiketMay1994,
bib:Maloney1994}. При этом границы применения метода существенно расширяются.
В частности, точечный генератор напряжения с заданным внутренним сопротивлением
оказывается весьма удобным для моделирования источников питания.

Достоинством этого дополнения является то, что при его использовании вид базовых
уравнений меняется весьма незначительно. Например,если источник ориентирован
вдоль оси~$z$, то в системе уравнений Макселла~\eqref{eq:MaxwellEquations}
изменится только одна составляющая векторного уравнения для ротора~$\vect{H}$:
\begin{equation}
    \label{eq:LumpedSource:MaxwellEquationsAmendment}
    (\rot\vect{H})_z = \epsilon\frac{\partial{E_z}}{\partial{t}} +
        \sigma E_z + \frac{I_L}{\Delta{x}\Delta{y}},
\end{equation}
где $I_L$ --- ток через источник.

Подвернув уравнение~\eqref{eq:LumpedSource:MaxwellEquationsAmendment} действию
разностной схемы~\eqref{eq:FiniteDifferenceScheme} и полагая~$\sigma=0$ получим
следующую формулу:
\begin{multline}
    \label{eq:LumpedSource:FdtdEquationsAmendmentWithI}
    \fYee{E_z}{n+1}{i,j,k} = \fYee{E_z}{n}{i,j,k} +
        \frac{\Delta{t}}{\yee{\epsilon}{}{i,j,k}}
        \left[
            \frac{\yee{H_y}{n+1}{i,j,k}-\yee{H_y}{n+1}{i-1,j,k}}{\Delta{x}} -
            \frac{\yee{H_x}{n+1}{i,j,k}-\yee{H_x}{n+1}{i,j-1,k}}{\Delta{y}}
        \right] - \\ -
        \frac{\yee{I_L}{n+1}{}\Delta{t}}
             {\yee{\epsilon}{}{i,j,k}\Delta{x}\Delta{y}}.
\end{multline}

Согласно закону Ома, для рассматриваемого сосредоточенного генератора напряжения
с внутренним сопротивлением $R$ будет иметь место равенство
\begin{equation}
    \label{eq:LumpedSource:OhmLaw}
    \fYee{I_L}{n+1}{} = \frac{\Delta{z}}{2R}
    \left(
        \fYee{E_z}{n+1}{i,j,k} - \Yee{E_z}{n}{i,j,k}
    \right) +
    \frac{\yee{V_s}{n+1}{}}{R},
\end{equation}
где $V_s$ --- генерируемое источником напряжение.

После подстановки~\eqref{eq:LumpedSource:OhmLaw}
в~\eqref{eq:LumpedSource:FdtdEquationsAmendmentWithI} получим простое уравнение
для~$E_z$:
\begin{multline}
    \label{eq:LumpedSource:FdtdEquationsAmendmentWithU}
    \fYee{E_z}{n+1}{i,j,k} =
        \fYee{C_E}{}{i,j,k}\fYee{E_z}{n}{i,j,k}~~+~~
        \Yee{C_H}{}{i,j,k}
        \left[
            \frac{\yee{H_y}{n+1}{i,j,k}-\yee{H_y}{n+1}{i-1,j,k}}{\Delta{x}}
        \right. - \\ -
        \left.
            \frac{\yee{H_x}{n+1}{i,j,k}-\yee{H_x}{n+1}{i,j-1,k}}{\Delta{y}} +
            \frac{\yee{V_s}{n+1}{}}{R\Delta{x}\Delta{y}}
        \right],
\end{multline}
где $C_E$ и~$C_H$ определяются по формулам:
\begin{equation}
    \newcommand\XA{\displaystyle
        \frac{\Delta{t}}{\yee{\epsilon}{}{i,j,k}}}
    \newcommand\XB{\displaystyle
        \frac{\Delta{t}\Delta{z}}{2R\yee{\epsilon}{}{i,j,k}\Delta{x}\Delta{y}}}
    % --
    \fYee{C_E}{}{i,j,k} = \frac{1-\XB}{1+\XB}, \quad
    \fYee{C_H}{}{i,j,k} = \frac{\XA}{1+\XB}.
\end{equation}
