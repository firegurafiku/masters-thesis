
\section{Вывод уравнений профиля антенны}

Получим выражения, описывающие профиль рупорных антенн с~экспоненциальным
и~линейным раскрывами. Прежде всего введем систему прямоугольных координат.
Пусть антенна расположена симметрично относительно оси~$O_x$ и~главное
направление излучения антенны совпадает с~направлением этой оси, входу антенны
соответствует координата $x=0$. Лепестки антенны расположены симметрично
друг другу относительно плоскости $O_{xy}$. Тогда можем записать выражения для
расстояния между лепестками:
\begin{align}
	\label{eq:AntennaProfiles:Linear}
    H(x) &= H_0 + (H_R-H_0)\frac{x}{R}, && \text{(линейный раскрыв)} \\
    \label{eq:AntennaProfiles:Exponential}
    H(x) &= H_0 \exp
    \left[
        \left(\frac{x}{R}\right)^\alpha\ln\frac{H_R}{H_0}
    \right] && \text{(экспоненциальный раскрыв~\cite{bib:Karshenas2009})}
\end{align}

\noindent
В этих формулах:
\begin{where}
\item $H_0$ --- расстояние между лепестками в~начале раскрыва;
\item $H_R$ --- расстояние между лепестками в~конце раскрыва;
\item $R$ --- длина раскрыва антенны вдоль координаты~$x$.
\end{where}

\noindent
Будем считать антенну длинной линией с~распределенными параметрами, фактически
--- полосковой линией с~переменными шириной и~расстоянием между проводниками.
В~\cite{bib:Aizenberg1985} показано, что уровень отражения входного сигнала
в~длинной линии получается наименьшим, если ее волновое сопротивление изменяется
в~пространстве по~экспоненциальному закону от~входного сопротивления антенны
до~сопротивления нагрузки на~выходе антенны. Итак, волновое сопротивление антенн
(с~обоими рассматриваемыми типами раскрыва) имеет вид:
\begin{equation}
    \label{eq:AntennaProfiles:HornWaveResistance}
    Z(x) = Z_0 \exp
    \left[
        \left(\frac{Z_R}{Z_0}\right)
        \frac{x}{R}
    \right],
\end{equation}
где $Z_0$ --- входное сопротивление антенны, $Z_R$ --- сопротивление нагрузки,
на~которую выполняется согласование.

Теперь воспользуемся известным соотношением для волнового сопротивления
полосковой линии:
\begin{equation}
    \label{eq:AntennaProfiles:MicrostripWaveResistance}
    Z = \frac{H\zeta_0}{B},
\end{equation}
%
\begin{where}
\item $H$ --- расстояние между проводниками линии;
\item $B$ --- ширина проводников;
\item $\zeta_0=120\pi$ --- т.~н. волновое сопротивление вакуума.
\end{where}

\noindent
В~нашей задаче $H$ и~$B$ не~являются постоянными, кроме того, нас интересует
значение ширины лепестка антенны, поэтому выразим его из
формулы~\eqref{eq:AntennaProfiles:MicrostripWaveResistance}:
\begin{equation}
    \label{eq:AntennaProfiles:WidthFromWaveResistance}
    B(x) = \frac{H(x)\zeta_0}{Z(x)}.
\end{equation}

Подставляя~\eqref{eq:AntennaProfiles:Linear} и~\eqref{eq:AntennaProfiles:Exponential}
в~\eqref{eq:AntennaProfiles:WidthFromWaveResistance}, получим после несложных
преобразований окончательные выражения для ширины лепестков антенн:
\begin{align}
    % --
    \label{eq:AntennaProfiles:LinearWidth}
    B(x) &= \frac{\zeta_0}{Z_0}
    \left[ H_0 + (H_R-H_0)\frac{x}{R} \right]
    \exp\left[
        -\frac{x}{R}\ln\frac{Z_R}{Z_0}
    \right] & \text{(линейный)} \\
    % --
    \label{eq:AntennaProfiles:ExponentialWidth}
    B(x) &= \frac{H_0\zeta_0}{Z_0}
    \exp\left[
        \left(\frac{x}{R}\right)^\alpha \ln\frac{H_R}{H_0} -
        \frac{x}{R} \ln\frac{Z_R}{Z_0}
    \right] & \text{(экспоненциальный)}
\end{align}

Это и~есть искомые профили антенн. Графики этих зависимостей при различных
значениях параметра~$Z_R$ изображены на~рис.~\ref{fig:ProfileWidths}
в~разделе~\ref{div:DirectionalPatternExperimentDescription}.
