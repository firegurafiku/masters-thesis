

\section{Решение задач на открытых областях}
\label{section:GridReduction}

Приведенные здесь базовый алгоритм в неизменном виде пригоден только в случае
бесконечной счетной области. Очевидно, этот случай невозможно осуществить на
практике (по крайней мере, за конечное время). Поэтому были разработаны методы,
позволяющие получать решение, близкое к теоретическому, при ограниченном счетном
объеме. Это возможно благодаря дополнительной обработке полей на границе
счетного объема.


\subsection{Методы сокращения сетки}

В настоящее время наиболее широко используемые методы ограничения счетной
области. Основными такими методами являются:
\begin{itemize}
    \item граничное условие отражения;
	\item граничные условия поглощения;
	\item системы идеально согласованных слоев (PML).
\end{itemize}

Использование первого из этих методов приводит к многократному переотражению
сигнала внутри счетной области, два других позволяют уменьшить отражение волны
от границы. Условия поглощения намного проще, чем PML, однако, система
согласованных слоев позволяет получить на порядки меньшие по величине
коэффициенты отражения от границы.

Понятие идеально согласованных слоев (PML) было введено Жаном-Пьером Беренже
в~\cite{bib:Berenger1994}. Позже оригинальная методика Беренже была
усовершенствована и дополнена, в результате были разработаны методы
UPML (uniaxial PML), CPML (convolutional PML), а также PML высших порядков.
Два последних метода имеют повышенную способность к поглощению падающих волн
и принципиально могут быть помещены ближе к изучаемой рассеивающей или
излучающей структуре, чем оригинальные PML Беренже.

В данной работе ниже мы подробно рассмотрим условия PML и определим их основные
характеристики.


\subsection{Условие идеального отражения}

Предположим, что за счетным объемом напряженность поля равна нулю. Это означало
бы, что счетный объем окружен бесконечным идеальным проводником, а значит, все
волны полностью отразятся назад внутрь счетного объема. Хотя в природе не
существует идеальных проводников, это условие может быть полезным в ряде
случаев.

Так как волны отражаются внутрь счетного объема, это значит, что он --- счетный
объем --- должен быть достаточно большим, чтобы границы находились достаточно
далеко от рассчитываемой структуры и отраженная от границ волна не успевала
существенно повлиять на физические процессы. Однако, это приводит к серьезному
увеличению времени расчета и объема необходимой для моделирования памяти.

На практике это условие реализуется очень просто: счетный объем окружается
дополнительным слоем ячеек, поле в которых никогда не высчитывается и изначально
инициализируется нулями.


\subsection{Условия поглощения}

Наличие отражения волн внутрь счетного объема в большинстве случаев искажает
картину распространения сигнала. С этим можно бороться при помощи специальных
граничных условий, которые поглощали бы падающие волны, не давая им отразиться
от границы объема.

Условия поглощения (англ.~absorbtion conditions, ABC) были разработаны для
моделирования границы счетного объема, поглощающей падающие на нее волны. Все
эти условия используют значения полей в смежных ячейках и в предыдущие моменты
времени для того, чтобы получить текущее значение поля на границе.

Всего известно достаточно много граничных условий поглощения, однако некоторые
из них не могут быть применены в прямоугольных декартовых координатах, другие
требуют для работы слишком громоздких вычислений. Поэтому реально применяются
граничные условия только двух типов: граничные условия Ляо и Мура. Так как на
практике эти граничные условия применяются намного реже, чем PML, в данной
работе они не будут рассматриваться.


\subsection{Граничные условия PML}

Данный тип граничных условий (строго говоря, являющихся поглощающей приграничной
областью, а не граничным условием, как таковым) обладает одними из лучших
характеристик. При использовании PML предполагается, что счетную область
окружается слоем анизотропного поглощающего материала, обладающий рядом свойств,
подробно описанных ниже.

\paragraph*{Волновое сопротивление.}
На границе раздела «счетная область --- согласованный слой» волновое
сопротивление не изменяется, что достигается выполнением следующего соотношения:
%%
\begin{equation}
\label{eq:EpsilonAndMu}
\frac{\sigma}{\varepsilon} = \frac{\sigma^*}{\mu}.
\end{equation}

\paragraph*{Профиль потерь.}
Электрическая и магнитная проводимость возрастают с увеличением расстояния
вглубь слоя PML по некоторому закону, называемому \emph{профилем потерь}.

В оригинальной работе Беренже был предложен следующий профиль потерь, как
оптимальный по уровню отражения от границы
раздела~\cite{bib:Berenger1994,bib:Berenger1996}:
%%
\begin{itemize}

\item для электрической проводимости:
\begin{align}
\label{eq:LossProfile}
\sigma_\text{PML}(i) &= \left\{
\begin{array}{ll}
    \sigma_\text{PML0}\frac{\sqrt{g}-1}{\ln{g}}     & \text{при~} i=0,\\
    \sigma_\text{PML0}\frac{g-1}{\sqrt{g}\ln{g}}g^i & \text{при~} i=\frac12,1,\frac32,~\ldots
\end{array}
\right.\\
\sigma_\text{PML0} &= -\frac{\epsilon_0 \ln{g}} {2 d \epsilon_0\mu_0 (g^N-1)} \ln{r};
% NOTE: Здесь применена особая уличная магия.
\intertext{\item для магнитной проводимости:}
%\begin{align}
    \sigma^{*}_\text{PML}(i) &= \frac{\mu}{\epsilon} \sigma_\text{PML}(i).
\end{align}

\end{itemize}

\noindent
В приведенной выше формуле фигурируют следующие величины:
\begin{where}
\item $g$ --- параметр, подбираемый эмпирически;
\item $r$ --- требуемый коэффициент отражения;
\item $d$ --- интервал дискретизации пространства в данном направлении;
\item $N$ --- толщина слоя PML.
\end{where}

\noindent
Приблизительная оценка толщины слоя PML может быть получена из следующего
соотношения:
\begin{equation}
\label{eq:hz1}
    N_\text{min}= \frac{1}{\ln g}
    \left[
        1 - \frac{\Theta t_p(\sqrt{g}-1)}{4\pi d \sqrt{\epsilon_0\mu_0}} \ln{r}
    \right],
\end{equation}
где $t_p$ --- длительность процесса моделирования во времени,
а параметр~$\Theta$ подбираются эмпирически (обычно $\Theta\approx10$).

\paragraph*{Частота отсечки.}
При использовании системы идеально согласованных слоев область допустимых частот
становится ограниченной снизу. Сигналы с частотой, меньшей чем некоторое
значение $f_{min}$ (называемое \emph{частотой отсечки}), отражаются от границы
PML. Ниже приведено выражение для частоты отсечки:
\begin{equation}
    \label{eq:PmlCutoffFrequency}
    f_{min} = \frac{\sigma_\text{PML}(0)}{2\pi\epsilon_0}.
\end{equation}


\subsection{Базовые уравнения PML}

Основная идея получения базовых уравнений для расчета компонентов поля
в согласованном слое состоит в том, что компоненты векторов $\vect{E}$
и $\vect{H}$ представляются в виде суммы двух слагаемых:
\begin{equation}
\label{eq:PmlSplitFieldEquations}
\left\{
    \begin{aligned}
        E_x = E_{xy}+E_{xz}, \\
        E_y = E_{yx}+E_{yz}, \\
        E_z = E_{zx}+E_{zy}.
    \end{aligned}
\right.\quad
\left\{
    \begin{aligned}
        H_x = H_{xy}+H_{xz}, \\
        H_y = H_{yx}+H_{yz}, \\
        H_z = H_{zx}+H_{zy}.
    \end{aligned}
\right.
\end{equation}

% TODO: Доверстать этот пиздец.
% \begin{пиздец}

\begin{eqnarray}
\label{eq:pml13}
%---
\left\{
\begin{array}{rcl}
  %--- xy
  \varepsilon \frac{\partial}{\partial t}E_{xy} + \sigma_y E_{xy} & = & \frac{\partial}{\partial y}(H_{zx}+H_{zy}) \\
  %--- xz
  \varepsilon \frac{\partial}{\partial t}E_{xz} + \sigma_z E_{xz} & = & \frac{\partial}{\partial z}(H_{yz}+H_{yx}) \\
  %--- yz
  \varepsilon \frac{\partial}{\partial t}E_{yz} + \sigma_z E_{yz} & = & \frac{\partial}{\partial z}(H_{xy}+H_{xz}) \\
  %--- yx
  \varepsilon \frac{\partial}{\partial t}E_{yx} + \sigma_x E_{yx} & = & \frac{\partial}{\partial x}(H_{zx}+H_{zy}) \\
  %--- zx
  \varepsilon \frac{\partial}{\partial t}E_{zx} + \sigma_x E_{zx} & = & \frac{\partial}{\partial x}(H_{yx}+H_{yz}) \\
  %--- zy
  \varepsilon \frac{\partial}{\partial t}E_{zy} + \sigma_y E_{zy} & = & \frac{\partial}{\partial y}(H_{xy}+H_{xz}) \\
\end{array}
\right.
  \\ %-------------------------------
\left\{
\begin{array}{rcl}
  %--- xy
  \varepsilon \frac{\partial}{\partial t}H_{xy} + \sigma_y^* H_{xy} & = & \frac{\partial}{\partial y}(E_{zx}+E_{zy}) \\
  %--- xz
  \varepsilon \frac{\partial}{\partial t}H_{xz} + \sigma_z^* H_{xz} & = & \frac{\partial}{\partial z}(E_{yz}+E_{yx}) \\
  %--- yz
  \varepsilon \frac{\partial}{\partial t}H_{yz} + \sigma_z^* H_{yz} & = & \frac{\partial}{\partial z}(E_{xy}+E_{xz}) \\
  %--- yx
  \varepsilon \frac{\partial}{\partial t}H_{yx} + \sigma_x^* H_{yx} & = & \frac{\partial}{\partial x}(E_{zx}+E_{zy}) \\
  %--- zx
  \varepsilon \frac{\partial}{\partial t}H_{zx} + \sigma_x^* H_{zx} & = & \frac{\partial}{\partial x}(E_{yx}+E_{yz}) \\
  %--- zy
  \varepsilon \frac{\partial}{\partial t}H_{zy} + \sigma_y^* H_{zy} & = & \frac{\partial}{\partial y}(E_{xy}+E_{xz}) \\
\end{array}
\right.
\end{eqnarray}

Анизотропия согласованного слоя проявляется в следующем: каждой ячейке соответствует набор $(\sigma_x,\sigma_y,\sigma_z)$, причем элемент $\sigma_\alpha$ ($\alpha={x,y,z}$) не равен нулю только для ячеек слоя, пересекаемого соответствующей осью координат.


В дискретной форме уравнения (\ref{eq:pml13}) выглядят следующим образом:
\begin{equation}
\left\{
\begin{aligned}
H_{xy (i,j,k)}^{n+1} = D_{ay (i,j,k)} H_{xy (i,j,k)}^{n} - D_{by (i,j,k)}
\left[
    \frac{E_{z (i,j+1,k)}^n - E_{z (i,j,k)}^n}{\Delta y}
\right] \\
%---
H_{xz (i,j,k)}^{n+1} = D_{az (i,j,k)} H_{xz (i,j,k)}^{n} - D_{bz (i,j,k)}
\left[
    \frac{E_{y (i,j,k+1)}^n - E_{y (i,j,k)}^n}{\Delta z}
\right] \\
%---
H_{yx (i,j,k)}^{n+1} = D_{ax (i,j,k)} H_{yx (i,j,k)}^{n} - D_{bx (i,j,k)}
\left[
    \frac{E_{z (i+1,j,k)}^n - E_{z (i,j,k)}^n}{\Delta x}
\right] \\
%---
H_{yz (i,j,k)}^{n+1} = D_{az (i,j,k)} H_{yz (i,j,k)}^{n} - D_{bz (i,j,k)}
\left[
    \frac{E_{x (i,j,k+1)}^n - E_{x (i,j,k)}^n}{\Delta z}
\right] \\
%---
H_{zx (i,j,k)}^{n+1} = D_{ax (i,j,k)} H_{zx (i,j,k)}^{n} - D_{bx (i,j,k)}
\left[
    \frac{E_{y (i+1,j,k)}^n - E_{z (i,j,k)}^n}{\Delta x}
\right] \\
%---
H_{zy (i,j,k)}^{n+1} = D_{ay (i,j,k)} H_{zy (i,j,k)}^{n} - D_{by (i,j,k)}
\left[
    \frac{E_{x (i,j+1,k)}^n - E_{x (i,j,k)}^n}{\Delta y}
\right]
\end{aligned}
\right.
\end{equation}

\begin{equation}
\left\{
\begin{aligned}
E_{xy (i,j,k)}^{n+1} = C_{ay (i,j,k)} E_{xy (i,j,k)}^{n} - C_{by (i,j,k)}
\left[
    \frac{H_{z (i,j,k)}^{n+1} - H_{z (i,j-1,k)}^{n+1}}{\Delta y}
\right] \\
%---
E_{xz (i,j,k)}^{n+1} = C_{az (i,j,k)} E_{xz (i,j,k)}^{n} - C_{bz (i,j,k)}
\left[
    \frac{H_{y (i,j,k)}^{n+1} - H_{y (i,j,k-1)}^{n+1}}{\Delta z}
\right] \\
%---
E_{yx (i,j,k)}^{n+1} = C_{ax (i,j,k)} E_{yx (i,j,k)}^{n} - C_{bx (i,j,k)}
\left[
    \frac{H_{z (i,j,k)}^{n+1} - H_{z (i-1,j,k)}^{n+1}}{\Delta x}
\right] \\
%---
E_{yz (i,j,k)}^{n+1} = C_{az (i,j,k)} E_{yz (i,j,k)}^{n} - C_{bz (i,j,k)}
\left[
    \frac{H_{x (i,j,k)}^{n+1} - H_{x (i,j,k-1)}^{n+1}}{\Delta z}
\right] \\
%---
E_{zx (i,j,k)}^{n+1} = C_{ax (i,j,k)} E_{zx (i,j,k)}^{n} - C_{bx (i,j,k)}
\left[
    \frac{H_{y (i,j,k)}^{n+1} - H_{z (i-1,j,k)}^{n+1}}{\Delta x}
\right] \\
%---
E_{zy (i,j,k)}^{n+1} = C_{ay (i,j,k)} E_{zy (i,j,k)}^{n} - C_{by (i,j,k)}
\left[
    \frac{H_{x (i,j,k)}^{n+1} - H_{x (i,j-1,k)}^{n+1}}{\Delta y}
\right]
\end{aligned}
\right.
\end{equation}

Коэффициенты $C$ и $D$ находятся по формулам:
\begin{equation*}
\begin{aligned}
D_{ax (i,j,k)} =
\frac
{
    1-\frac{\sigma_{x (i,j,k)}^*\Delta t}{2\mu_{(i,j,k)}}
}{
    1+\frac{\sigma_{x (i,j,k)}^*\Delta t}{2\mu_{(i,j,k)}}
}, \\
%---
D_{bx (i,j,k)} =
\frac
{
    \frac{\Delta t}{\mu_{(i,j,k)}}
}{
    1+\frac{\sigma_{x (i,j,k)}^*\Delta t}{2\mu_{(i,j,k)}}
},
\end{aligned}
%---------------
\quad
\begin{aligned}
D_{ay (i,j,k)} =
\frac
{
    1-\frac{\sigma_{y (i,j,k)}^*\Delta t}{2\mu_{(i,j,k)}}
}{
    1+\frac{\sigma_{y (i,j,k)}^*\Delta t}{2\mu_{(i,j,k)}}
}, \\
%---
D_{by (i,j,k)} =
\frac
{
    \frac{\Delta t}{\mu_{(i,j,k)}}
}{
    1+\frac{\sigma_{y (i,j,k)}^*\Delta t}{2\mu_{(i,j,k)}}
},
\end{aligned}
%---------------
\quad
\begin{aligned}
D_{az (i,j,k)} =
\frac
{
    1-\frac{\sigma_{z (i,j,k)}^*\Delta t}{2\mu_{(i,j,k)}}
}{
    1+\frac{\sigma_{z (i,j,k)}^*\Delta t}{2\mu_{(i,j,k)}}
}, \\
%---
D_{bz (i,j,k)} =
\frac
{
    \frac{\Delta t}{\mu_{(i,j,k)}}
}{
    1+\frac{\sigma_{z (i,j,k)}^*\Delta t}{2\mu_{(i,j,k)}}
},
\end{aligned}
%---------------
\end{equation*}

\begin{equation*}
\begin{aligned}
D_{ax (i,j,k)} =
\frac
{
    1-\frac{\sigma_{x (i,j,k)}\Delta t}{2\mu_{(i,j,k)}}
}{
    1+\frac{\sigma_{x (i,j,k)}\Delta t}{2\mu_{(i,j,k)}}
}, \\
%---
D_{bx (i,j,k)} =
\frac
{
    \frac{\Delta t}{\mu_{(i,j,k)}}
}{
    1+\frac{\sigma_{x (i,j,k)}\Delta t}{2\mu_{(i,j,k)}}
},
\end{aligned}
%---------------
\quad
\begin{aligned}
D_{ay (i,j,k)} =
\frac
{
    1-\frac{\sigma_{y (i,j,k)}\Delta t}{2\mu_{(i,j,k)}}
}{
    1+\frac{\sigma_{y (i,j,k)}\Delta t}{2\mu_{(i,j,k)}}
}, \\
%---
D_{by (i,j,k)} =
\frac
{
    \frac{\Delta t}{\mu_{(i,j,k)}}
}{
    1+\frac{\sigma_{y (i,j,k)}\Delta t}{2\mu_{(i,j,k)}}
},
\end{aligned}
%---------------
\quad
\begin{aligned}
D_{az (i,j,k)} =
\frac
{
    1-\frac{\sigma_{z (i,j,k)}\Delta t}{2\mu_{(i,j,k)}}
}{
    1+\frac{\sigma_{z (i,j,k)}\Delta t}{2\mu_{(i,j,k)}}
}, \\
%---
D_{bz (i,j,k)} =
\frac
{
    \frac{\Delta t}{\mu_{(i,j,k)}}
}{
    1+\frac{\sigma_{z (i,j,k)}\Delta t}{2\mu_{(i,j,k)}}
},
\end{aligned}
%---------------
\end{equation*}

% \end{пиздец}

Следует заметить, что в реальной ситуации всегда присутствуют
отражения~\cite{bib:Berenger1994,bib:Berenger1996}.
Они складываются из:
\begin{itemize}
  \item отражений от первого слоя PML;
  \item отражений между слоями PML;
  \item отражений от проводящей границы за последним слоем PML.
\end{itemize}

\noindent
В связи с этим, для уменьшения отражения от первого слоя значение~$\sigma_0$
специально выбирается небольшим. Отражения между слоями подавляются за счет
ограничения скорости роста профиля потерь. Для уменьшения влияния волны,
отраженной от бесконечно проводящей границы, увеличивается число слоев
PML~\cite{bib:Berenger1996}.
