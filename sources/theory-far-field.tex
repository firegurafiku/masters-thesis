

\section{Нахождение поля в дальней зоне}

Описанные в разделе~\ref{section:CoreMethod} работы методы численного решения
уравнений Максвелла вместе с описанными в разделе~\ref{section:GridReduction}
методами решения задач на открытых обласях дают возможность вычисления
электромагнитного поля внутри некоторой ограниченной области пространства.
Однако на размер этой области накладываются ограничения, вызванные конечной
вычислительной мощностью современных ЭВМ, а~именно тремя основными факторами:
\begin{itemize}
\item ограниченной скоростью выполнения арифметических операций;
\item ограниченным объемом оперативной памяти;
\item ограниченной скоростью обмена данными между процессором
      и оперативной памятью.
\end{itemize}
Однако, существует ряд задач, в~которых необходим расчет электромагнитных полей
на~больших расстояниях (в дальней зоне) от~некоторого объекта, излучающего или
рассевающего электромагнитное поле. В~число этих задач входит, например,
получение диаграмм направленности антенн.

Для решения подобных задач существует эффективный способ вычисления полей
в~дальней зоне с~использованием результатов вычисления поля в ближней зоне,
выполненного методом FDTD. Этот способ заключается в~использовании
поверхностного интеграла Кирхгоффа.

Интеграл Кирхгофа связывает поле внутри ограниченного объема с~полем и~его
производными на~поверхности, ограничивающей объем. Эта формула была выведена
в~середине XIX~века немецким физиком Гюставом~Кирхгофом, и~во~временной области
имеет следующий вид:
\begin{equation}
    \label{eq:KirchhoffIntegral}
    \Psi(\vect{p},t) = \frac{1}{4\pi} \oint_\Gamma
    \left[
        \frac{1}{R}\nabla'\Psi(\vect{p'},t') -
        \frac{\vect{R}}{R^3}\Psi(\vect{p'},t') -
        \frac{\vect{R}}{cR^2}\frac{\partial}{\partial t'}\Psi(\vect{p'},t')
    \right]_\text{ret} \!\!\!\!\!\vect{n} dA'.
\end{equation}

\noindent
В этом выражении используются обозначения:
\begin{where}
\item $\Psi$ --- любая из шести компонент поля;
\item $\Gamma$ --- поверхность, ограничивающая объем в ближней зоне;
\item $dA'$ --- площадь элемента поверхности;
\item $\vect{n}$ --- единичный вектор нормали к~поверхности;
\item $\vect{p}$ --- точка наблюдения в дальней зоне;
\item $\vect{p'}$ --- точка на~поверхности;
\item $R$ --- расстояние между ними, $\vect{R} = \vect{p}-\vect{p'}$;
\item $c$ --- скорость света в вакууме.
\end{where}

\noindent
В~формуле~\eqref{eq:KirchhoffIntegral} индекс «ret» обозначает тот факт, что
интегрирование осуществляется с~учетом запаздывающего времени~$t'=t-R/c$. Также
отметим, что единичный вектор нормали~$\vect{n}$ всегда направлен внутрь
рассматриваемого замкнутого объема.

Формула~\eqref{eq:KirchhoffIntegral} выражает принцип Гюйгенса, согласно
которому каждая точка на волновом фронте служит фиктивным источником
воображаемой сферической волны. Каждый участок поверхности~$dA'$ излучает волну,
которая приходит в~точку наблюдения с~задержкой~$R/c$. При этом на каждом шаге
FDTD по времени на поверхности интегрирования возникает совокупность фиктивных
источников, поле от которых придет в~точку наблюдения с~разным запаздыванием,
поскольку расстояние~$R$ для всех точек различно. Это означает, что на одном
временном шаге находятся вклад элемента площади~$dA'$ в~разные временные участки
выходного сигнала в~точке наблюдения.

Шаг по времени при вычислении интеграла Кирхгофа тесно связан с~шагом по времени
FDTD и равен ему. Выходная последовательность в~точке наблюдения имеет такой же
шаг по времени. Однако величина~$R/c$ может не быть кратной шагу по времени,
поэтому получаемое время задержки округляется до ближайшего значения, кратного
шагу FDTD. Возникающая при этом ошибка незначительна, т.к. временной шаг
в~классическом FDTD мал по сравнению с~периодом колебаний вычисляемого сигнала.

Выражение~\eqref{eq:KirchhoffIntegral} в~такой форме неудобно для совместного
применения с~методом FDTD. Вывод формул, используемых на~практике, в~данной
работе не приводится и может быть найден, например, в~\cite{bib:Zelenin2006}.
Отметим также, что для получения лучших результатов поверхность интегрирования
следует располагать не в~непосредственной близости от исследуемого объекта, а~на
некотором отдалении от него (порядка 10~шагов дискретизации по~пространству).
