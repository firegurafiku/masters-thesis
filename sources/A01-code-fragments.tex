
\chapter{Ключевые фрагменты программного кода}

Ниже приведем некоторые фрагменты кода, отвечающие как за сериализацию в целом,
так и за работу непосредственно с форматом HDF~5 и использование
библиотеки \code{libhdf5}. В приложении представлен только код, отвечающий
за запись, т.~к. соответствующий код чтения во многом ему симметричен и может
быть опущен без ущерба для восприятия общей картины. Кроме того, опущены
реализации классов \code{HierarchyCrawler} и \code{SequentalWriter}, т.~к.
реализация первого достаточно объемна, а реализация второго --- практически
тривиальна.

\singlespacing

\subsubsection{HierarchyCrawler.h}
\lstinputlisting[language=c++]{appendix/HierarchyCrawler.h}

\subsubsection{SequentalWriter.h}
\lstinputlisting[language=c++]{appendix/SequentalWriter.h}

\subsubsection{HierarchyCrawlerImpl.h}
\lstinputlisting[language=c++]{appendix/HierarchyCrawlerImpl.h}

\subsubsection{SequentalWriterImpl.h}
\lstinputlisting[language=c++]{appendix/SequentalWriterImpl.h}

\subsubsection{HdfWriterImpl.h}
\lstinputlisting[language=c++]{appendix/HdfWriterImpl.h}

\subsubsection{HdfWriterImpl.cpp}
\lstinputlisting[language=c++]{appendix/HdfWriterImpl.cpp}
