\documentclass[a4paper,14pt,master]{disser}

\bibliographystyle{gost780u}
%\biboptions{compress}


\usepackage{mathtext}
%\usepackage{cmap}
% Русификация и utf-икация латеха
\usepackage[T2A]{fontenc}
\usepackage[utf8]{inputenc}
\usepackage[russian]{babel}

% Прочие пакеты и их настройки
\usepackage{amsmath}
\usepackage{amsfonts}
\usepackage{amssymb}
\usepackage{wasysym}  % for \oiint

\usepackage{graphicx}
\graphicspath{{../}}

\usepackage{verbatim}

\usepackage{tabularx}
\usepackage{multirow}

\usepackage{listings}
\lstloadlanguages{C++}
\lstset{language=C++,
        extendedchars=\true,
		basicstyle=\ttfamily\fontsize{11pt}{11pt}\selectfont}

\clubpenalty=10000
\widowpenalty=10000

\usepackage{geometry}
\geometry{
    left=2.5cm,
    right=1.5cm,
    top=1.5cm,
    bottom=1.0cm,
    footskip=1cm,
    includefoot}

\linespread{1.25}
%\onespacing
%\onehalfspacing



\parindent=1.1cm

\usepackage{url}

\setcounter{tocdepth}{1}

\usepackage{enumitem}
\newlist{where}{itemize}{1}
\newlist{chronology}{description}{1}
\setenumerate{
    leftmargin=\parindent,
    topsep=0cm,
    itemsep=0cm,
    parsep=0cm,
    partopsep=0cm}
\setitemize{
    label={--},
    leftmargin=\parindent,
    topsep=0cm,
    itemsep=0cm,
    parsep=0cm,
    partopsep=0cm}
\setdescription{
    topsep=0cm,
    leftmargin=\parindent,
    itemsep=0cm,
    parsep=0cm,
    partopsep=0cm}
\setlist[where]{
    label=,
    leftmargin=\parindent,
    topsep=0cm,
    itemsep=0cm,
    parsep=0cm,
    partopsep=0cm}
\setlist[chronology]{
    font=\normalfont,
    leftmargin=2.0cm,
    style=sameline,
    topsep=0.35cm,
    itemsep=0cm,
    parsep=0cm,
    partopsep=0cm}

%\renewcommand\chapter\addsectionw

%\usepackage{float}
%\newfloat{floatequation}{p}{floatequation.aux}{

\usepackage{calc}

\usepackage{numprint}

\usepackage{underscore}
\usepackage{listings}


\reversemarginpar


\allowdisplaybreaks

\renewcommand\epsilon\varepsilon
\renewcommand\phi\varphi
\renewcommand\le\leqslant
\renewcommand\ge\geqslant

\newcommand{\vect}[1]{\ensuremath{\vec{#1\mathstrut}}}

\makeatletter
%%%
\renewcommand\tocprethechapter{}
\renewcommand\tocpostthechapter{.}
\renewcommand\thechapteralign{}
\renewcommand\thechapterfont{\normalfont\bfseries}
\renewcommand\prethechapter{}
\renewcommand\postthechapter{.}
\renewcommand\tocchapterfont{\normalfont}
\renewcommand\tocchapterfillfont{\normalfont}
\renewcommand\tocchapternumfont{\normalfont}

%\renewcommand\chapterindent{}
\renewcommand\chapteralign{}
\renewcommand\chapterfont{\bfseries}
\renewcommand\beforechapter{}
\renewcommand\afterchapter{\par\nobreak\vskip 20\p@}

% Размер всех заголовков должен быть одинаковым.
\renewcommand\sectionfont{\normalfont\bfseries}

% Уберем дурацкие отступы после \paragraph и \subparagraрh.
\renewcommand\afterparagraph{-1ex}
\renewcommand\aftersubparagraph{-1ex}


\newcommand\tocchapterindent{0em}
\newcommand\tocchapternameindent{1.6em}

\renewcommand\tocposttheappendix{.}

%\renewcommand*\chapter{\clearpage\@startsection{chapter}{0}}
\renewcommand\l@chapter{\@tocline{chapter}{0}}

% Подписи к картинкам и таблицам должны быть того же размера, что
% и основной текст.
\renewcommand*\captionfont{\normalfont}
\renewcommand*\captionlabelfont{\normalfont}

\makeatother

%\DeclareMathOperator\div{div}
\DeclareMathOperator\rot{rot}
\DeclareMathOperator\grad{grad}

\newcommand{\VSWR}{\text{КСВН}}

\newcommand{\val}[1]{\numprint{#1}}
\newcommand{\valu}[2]{\val{#1}\,\ensuremath{#2}}

\newcommand\yee[3]{\left.#1\right|^{#2}_{#3}}
\newcommand\Yee[3]{#1\Big|^{#2}_{#3}}
\newcommand{\hanglimitsoperator}[4][c]{\text{\makebox[\widthof{$#2$}][#1]{ $#2^{#3}_{#4}$ }}}

\newcommand\fyee[3]{\hanglimitsoperator[l]{\left.#1\right|}{#2}{#3}~~~}
\newcommand\fYee[3]{\hanglimitsoperator[l]{#1\Big|}{#2}{#3}~~~}
\newcommand\yeediff[5]{\frac{\yee{#1}{#2}{#3} - \yee{#1}{#2}{#4}}{\Delta{#5}}}

\newcommand\code[1]{\texttt{#1}}

%\int\flimits^A_B
%\makebox[\widthof{\int}]\int\limits^A_B

% That special are only for convenience of browsing with xdvi program.
\special{papersize=\the\paperwidth,\the\paperheight}
