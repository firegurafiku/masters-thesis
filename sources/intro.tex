%
%
%
\chapter* {Введение}
\addcontentsline {toc} {chapter} {Введение}

В~настоящее время все более актуальным становится исследование
сверхширокополосных радиосистем, использующих в качестве сигналов сверхкороткие
импульсы~(СКИ) субнаносекундной длительности. Среди антенн, пригодных для излучения
и приема СКИ сигналов, можно выделить ТЕМ-рупоры с переменным волновым
сопротивлением, которое на практике, как правило, изменяется от~\valu{50}{Ом}
в точке запитки до $120\pi$~Ом у раскрыва.

Анализу именно таких антенны посвящена данная работа. В ней дается описание
численных экспериментов по определению характеристик TEM-рупорных
антенн при излучении СКИ-сигналов с использованием метода конечных разностей
во~временной области (FDTD), описание принципов работы которого вместе
с основными дополнениями также приводится в работе.

Геометрия TEM-рупорных антенн сложна для аналитических вычислений, однако она
поддается анализу с помощью численных методов. Из существующих алгоритмов
численного решения уравнений Максвелла был выбран именно FDTD так, как он,
являясь временным методом (в отличие от частотных), позволяет:
\begin{itemize}
\item получить зависимости КСВН от частоты за один расчет с помощью быстрого
      преобразования Фурье от временных значений эквивалентного тока
      и напряжения;
\item получить форму сигнала в определенной точке пространства, что весьма
      полезно при анализе задач, связанных с~СКИ;
\item рассчитать величину электромагнитного поля вне счетной области, что необходимо
      для нахождения диаграмм направленности.
\end{itemize}

Зависимость КСВН от частоты в рабочем диапазоне удобно использовать для оценки
уровня согласования антенны. При этом важным параметром для излучения СКИ
является нижняя граничная частота антенны, в качестве критерия для определения
которой можно принять требование не превышения КСВН во всем диапазоне частот
спектра излучаемого сигнала некоторого заранее выбранного значения.
Из теории плавных переходов известно, что
увеличение длины перехода с заданным профилем изменения волнового сопротивления
уменьшает его нижнюю граничную частоту перехода, поэтому в работе исследовалось
влияние длины ТЕМ-рупорной антенны с заданным экспоненциальным профилем
волнового сопротивления на КСВН и нижнюю граничную частоту антенны. С помощью
метода FDTD было проведено моделирование синтезированных профилей во временной
области, были построены и проанализированы зависимости КСВН от частоты
в диапазоне \val{0.1}--\valu{10}{ГГц}.

Из-за большой ширины спектра СКИ-сигнала описание антенны
при помощи диаграмм направленности, используемых обычно в случае узкополосных
сигналов, не дает представления о форме и амплитуде СШП-импульса,
так как вид диаграммы направленности антенны
может кардинальным образом отличаться на разных частотах, входящих
в его спектр. Поэтому использование в этом случае
энергетических диаграмм направленности, выражающих полную энергию, пришедшуюся
на то или иное направление за все время излучения предпочтительнее.

Также в работе дается описание компьютерных программ, разработанных для того,
чтобы рассчитывать вышеописанные параметры во временной области методом FDTD.
