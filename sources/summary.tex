\chapter*{Заключение}
\addcontentsline{toc}{chapter}{Заключение}

В работе был рассмотрен метод конечных разностей во временной области, были
выведены его основные уравнения и указаны условия стабильности.
Кроме того, были указаны способы задания граничных условий и рассмотрены
системы идеально согласованных слоев. Хотя область применения метода довольно широка, особо
выигрышным представляется его использование при исследовании нестационарных
процессов --- например, электромагнитного поля антенн при возбуждении их
короткими импульсами.

Так, Было проведено моделирование трех ТЕМ-рупоров с~экспоненциальным профилем
изменения волнового
сопротивления, обладающих различной длиной. В~результате расчета были получены
зависимости КСВН от частоты в~диапазоне \val{0.1}--\valu{10}{ГГц}. При этом
нижняя граничная частота по уровню КСВН~4 уменьшается при увеличении длины
антенны. Полученная закономерность хорошо согласовывается с~выводами из теории
переходов с~плавным изменением волнового сопротивления. Следует отметить,
что увеличение длины ТЕМ-рупора на~\valu{3}{см}, что соответствует~20\% от
исходной длины~\valu{150}{мм}, позволило понизить нижнюю граничную частоту
приблизительно на~\valu{100}{МГц}, увеличение на~\valu{5}{см} (на~33\% от
исходной длины) --- на \valu{140}{МГц}, что представляет весьма важный
практический результат.

% -- Про диаграммы направленности
Также были промоделированы шестнадцать ТЕМ-рупоров с линейными
и экспоненциальными раскрывами. Полученные результаты моделирования
использовались для нахождения импульсных характеристик антенны и расчета ее
отклика на сигналы, форма которых отлична от используемой при фактическом
моделировании. В результате расчета были получены энергетические диаграммы
направленности излучения сигналов разной длительности для антенн с различными
профилями, выходными сопротивлениями (в целях краткости изложения, в работе
приведены диаграммы направленности только для двух антенн и трех сигналов).
Было установлено, что диаграммы направленности антенны с экспоненциальным
профилем раскрыва уже, чем у антенны с линейным раскрывом, в особенности при
излучении сигналов, пространственная длительность которых меньше размеров
антенны.

