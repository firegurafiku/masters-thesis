
\section[Энергетические диаграммы направленности]
        {Энергетические диаграммы направленности и~импульсные характеристики антенн}
\label{div:DirectionalPatternsTheory}

Известно, что описание излучения импульсов субнаносекундной длительности имеет
особенности, связанные с~частотным составом излучаемого сигнала и~соотношением
протяженностей антенны и~излучаемого импульса. Так, полоса частот в~несколько
гигагерц делает неудобными для анализа характеристик излучения антенны
традиционные диаграммы направленности, применяемые для узкополосных
сигналов~\cite{bib:Immoreev2002}. Одна и~та же антенна на~частотах, отличающихся
в~несколько раз, может иметь совершенно разные характеристики излучения.

Кроме того, пространственная длительность импульса может не~превышать
протяженности излучающей части антенны. В~подобных случаях форма сигнала может
существенно зависеть от~угла наблюдения. При этом форма сигнала в~заданной точке
пространства не~может быть описана только усилением в~заданном направлении.
Таким образом, для того чтобы описать процесс излучения импульса антенной,
необходимо несколько характеристик~\cite{bib:Immoreev2002,bib:Astanin1989}:
\begin{itemize}
\item
энергетические диаграммы направленности --- характеризуют усредненное
пространственное распределение энергии за~все время существования излучаемого
сигнала;
\item
набор импульсных характеристик антенны, которые позволяют оценить форму импульса
в~заданной точке пространства по~входному сигналу произвольной формы.
\end{itemize}

Для расчета энергетических диаграмм направленности использовались данные
моделирования поля в~дальней зоне. Так как энергии электрической и~магнитной
составляющих поля равны, усредненная энергия ТЕМ-волны за~время существования
импульса в~заданной точке пространства может быть рассчитана по~электрической
составляющей с~помощью следующего соотношения:
\begin{equation}
    \label{eq:AverageWaveEnergy}
    W(r,\theta,\phi) = C\hanglimitsoperator{\int}{R/c+T}{R/c}
    \left| E_\theta(r,\theta,\phi) \right|^2 dt,
\end{equation}
где $c$ --- скорость света в~свободном пространстве, $T$ --- длительность
сигнала в~заданной точке. Далее проводилась нормировка на~максимальное значение
(поэтому в~расчетах было принято, что~$C=1$). Как видно
из~\eqref{eq:AverageWaveEnergy}, расчет энергетических диаграмм направленности
связан с~оценкой электрической составляющей поля в~заданной точке пространства.
При этом вызывает интерес оценка характеристик антенны при излучении сигналов
различной формы. В~этом случае пересчет поля в~дальней зоне может быть
осуществлен без~полного электродинамического моделирования, которое для
крупногабаритных антенн может занять значительное время. Достаточно точно
оценить отклик в~любой точке пространства можно, зная импульсную характеристику
системы <<антенна–пространство>> для данных координат.

Так как антенна и~окружающая среда линейны, связь между сигналом, возбуждающим
антенну, и~полем в~заданной точке пространства можно записать в~виде:
\begin{equation}
    \label{eq:TangentialElectricField}
	E_\theta(r,\theta,\phi) = h(r,\theta,\phi) * s(t),
\end{equation}
где $h(r,\theta,\phi)$ --- импульсная характеристика, $s(t)$ --- сигнал на~входе
антенны, а~звездочкой обозначен оператор свертки.

Если импульсная характеристика в~заданной точке пространства известна, то
с~помощью соотношения~\eqref{eq:TangentialElectricField}, можно рассчитать
электрическую составляющую поля в~этой точке. Если же ее требуется найти, то
соотношение~\eqref{eq:TangentialElectricField} следует рассматривать как
линейное уравнение относительно $h(r,\theta,\phi)$. Проведя моделирование
с~тестовым сигналом~$s(t)$, можно рассчитать электрическую составляющую поля
в~заданной точке, и, решив~\eqref{eq:TangentialElectricField},
найти~$h(r,\theta,\phi)$.

Имея набор импульсных характеристик в~заданных точках пространства, можно
рассчитать энергетические диаграммы направленности для различных сигналов,
не~прибегая к~полному электродинамическому моделированию антенны.
