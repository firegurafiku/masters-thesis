
\noindent
{\newcolumntype{C}{>{\centering\arraybackslash}X}
\begin{tabularx}{\textwidth}{lCr}
УДК 621.396.67 & \textbf{Реферат} & Кретов\,П.\,А.
\end{tabularx}}
\vspace{10mm}

\noindent
Исследование TEM-рупорных антенн для излучения и приема импульсных СШП-сигналов
методом конечных разностей во~временной области.\\
Магистерская диссертация по направлению 010800 Радиофизика, Воронеж, ВГУ, 2012\,г.
--- 73\,с., илл.~20, библ.~28 назв.

\vspace{10mm}

\noindent
\textbf{Ключевые слова:}
    метод конечных разностей во временной области,
    TEM-рупорная антенна,
    сверхширокополосный сигнал,
    сверхкороткий импульс,
    граничная частота,
    согласование волнового сопротивления,
    коэффициент стоячей волны по напряжению,
    энергетическая диаграмма направленности.

\vspace{10mm}
\noindent
Рассматривалась задача моделирования TEM-рупорных антенн с разными формами
раскрыва для нахождения энергетических диаграмм антенны и коэффициента стоячей
волны по напряжению на входе антенны. Для расчета использовался метод конечных
разностей во временной области, разработаны компьютерные программы для
моделирования.
\clearpage
