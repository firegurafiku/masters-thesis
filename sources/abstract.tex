
\noindent
{\newcolumntype{C}{>{\centering\arraybackslash}X}
\begin{tabularx}{\textwidth}{lCr}
УДК 621.396.67 & \textbf{Реферат} & Кретов\,П.\,А.
\end{tabularx}}
\vspace{10mm}

НАЗВАНИЕ!!!!.\\
Магистерская диссертация по направлению 010800 Радиофизика, Воронеж, ВГУ, 2012\,г.
--- !!!\,с., илл. !!!, библ. !!! назв.

\vspace{10mm}

\noindent
\textbf{Ключевые слова:}
    метод конечных разностей во временной области,
    TEM-рупорная антенна,
    сверхширокополосный сигнал,
    сверхкороткий импульс,
    граничная частота,
    согласование волнового сопротивления,
    коэффициент стоячей волны по напряжению,
    энергетическая диаграмма направленности.

\vspace{10mm}
Рассматривалась задача моделирования TEM-рупорной антенны, нахождения
коэффициента стоячей волны по напряжению на входе антенны. Проводились
эксперименты по численному нахождению КСВН для антенн с различной геометрией.
Для расчета использовался метод конечных разностей во временной области.
\clearpage
