
\noindent
{\newcolumntype{C}{>{\centering\arraybackslash}X}
\begin{tabularx}{\textwidth}{lCr}
УДК 621.396.67 & \textbf{Реферат} & Кретов\,П.\,А.
\end{tabularx}}
\vspace{10mm}

Расчет значений нижней граничной частоты для TEM-рупоров различной длины
с использованием метода конечных разностей во временной области.\\
Магистерская диссертация по направлению 010800 Радиофизика, Воронеж, ВГУ, 2012\,г.
--- \lastpage\,с., илл. 11, библ. 21 назв.

\vspace{10mm}

\noindent
\textbf{Ключевые слова:} сверхширокополосный сигнал, сверхкороткий импульс, TEM-рупорная антенна, граничная частота, метод конечных разностей, согласование волнового сопротивления, коэффициент стоячей волны по напряжению.

\vspace{10mm}
Рассматривалась задача моделирования TEM-рупорной антенны, нахождения коэффициента стоячей волны по напряжению на входе антенны. Проводились эксперименты по численному нахождению КСВН для антенн с различной геометрией. Для расчета использовался метод конечных разностей во временной области.
\newpage

